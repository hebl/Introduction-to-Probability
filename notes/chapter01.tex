\chapter{样本空间与概率}

\section{集合}

\subsection{集合运算}

集合运算:交集、并集、补集

\subsection{集合代数}

\kw{德摩根定律 De Morgan's laws}

$$(\bigcup_n S_n)^c = \bigcap S_n^c$$ 

$$(\bigcap_n S_n)^c = \bigcup S_n^c$$ 

\section{概率模型}

\kw{样本空间} $\Omega$

离散

连续 $\Omega = \{(x,y) \mid 0 \leq x, y \leq 1 \}$ 

\subsection{公理}
 
 \kw{Probability Axioms}

 \begin{ol}
   \item $\RSP(A) \geq 0$
   \item $\RSP(\Omega) = 1$
   \item 如果 $A \cap B = \phi$,那么 $\RSP(A \cup B) = \RSP(A) + \RSP(B)$
 \end{ol}

 推论:

\begin{ul}
  \item $\RSP(A) \leq 1$
  \item $\RSP(\phi) = 0$
  \item $\RSP(A) + \RSP(A^c) = 1$
  \item 独立事件:$$\RSP(A_1 \cup A_2 \cup \cdots \cup A_n) = \sum_{i=1}^n \RSP(A_i)$$
  \item 如果 $A \subset B$,那么$\RSP(A) \leq \RSP(B)$
  \item $\RSP(A \cup B) = \RSP(A) + \RSP(B) - \RSP(A \cap B)$
  \item $\RSP(A \cup B) \leq \RSP(A) + \RSP(B)$
\end{ul}


\section{条件概率}

\kw{条件概率}:给定条件$B$发生之下$A$发生的概率。(\kt{这是定义,不是定理})

$$\RSP (\wblue{A} \mid \wred{B}) = \frac{\RSP(\wblue{A} \cap \wred{B})}{\RSP(\wred{B})}$$

当试验为有限集$\Omega$。那么:

$$\RSP (\wblue{A} \mid \wred{B}) = \frac{\kt{事件}(\wblue{A} \cap \wred{B})\kt{的试验结果数}}{\kt{事件B的试验结果数}}$$

条件概率也符合概率的公理:

\begin{ol}
   \item $\RSP(\wblue{A} \mid \wred{B}) \geq 0$
   \item $\RSP(\Omega \mid \wred{B}) = 1$
   \item $\RSP(\wred{B} \mid \wred{B}) = 1$
   \item 如果 $\wblue{A} \cap C = \phi$,那么 $\RSP(\wblue{A} \cup C \mid \wred{B}) = \RSP(\wblue{A} \mid \wred{B}) + \RSP(C \mid \wred{B})$
 \end{ol}

 乘法法则

 $$\RSP (\wblue{A} \mid \wred{B}) = \frac{\RSP(\wblue{A} \cap \wred{B})}{\RSP(\wred{B})}$$

\begin{align*}
 \RSP(\wblue{A} \cap \wred{B}) & = \RSP(\wred{B}) \RSP (\wblue{A} \mid \wred{B}) \\
 & = \RSP(\wblue{A}) \RSP (\wred{B} \mid \wblue{A})
\end{align*}

\section{全概率公式和贝叶斯准则}

\subsection{全概率公式}

 $$\RSP(\wred{B}) = \sum_i \RSP(\wblue{A_i})\RSP(\wred{B} \mid \wblue{A_i})$$

\subsection{贝叶斯准则}

%\begin{equation}
\begin{align*}
 \RSP(\wblue{A_i} \mid \wred{B}) & = \frac{\RSP(\wblue{A_i})\RSP(\wred{B} \mid \wblue{A_i})}{\RSP(\wred{B})} \\
 & = \frac{\RSP(\wblue{A_i})\RSP(\wred{B} \mid \wblue{A_i})}{\sum_j \RSP(\wblue{A_j})\RSP(\wred{B} \mid \wblue{A_j})}
 \end{align*}
% \notag
%\end{equation}

\kw{贝叶斯推理}

先验$\wred{A_i}$,后验$\wred{B}$

$$A_i \xrightarrow[\RSP(\wred{B} \mid \wblue{A_i})]{\kt{模型}} \wred{B}$$

$$\wred{B} \xrightarrow[\RSP(\wblue{A_i} \mid \wred{B})]{\kt{推理}} \wblue{A_i}$$

\section{独立性}

两个事件$A$和$B$相互独立,则满足:

$$\RSP(A \cap B) = \RSP(A)\RSP(B)$$

若$B$还满足$\RSP(B) > 0$,则独立性等价于

$$\RSP(A \mid B) = \RSP(A)$$

两个事件不相交不代表独立。不相交的事件 $\RSP(A \cap B) = 0 $,但$\RSP(A)\RSP(B) > 0$,显然这两个不是独立的。

\kw{性质}

\begin{ul}
  \item 若$A$和$B$相互独立,则$A$和$A^c$也相互独立。
  \item 条件独立:$\RSP(A \cap B \mid C) = \RSP(A \mid C)\RSP(B \mid C)$
  \item 独立性并不蕴含条件独立性,反之亦然。
\end{ul}

\subsection{条件独立}

若$\RSP(C) > 0$

$$\RSP(A \cap B \mid C) = \RSP(A \mid C)\RSP(B \mid C)$$

只要满足$\RSP(B \mid C) > 0$,则

$$\RSP(A \mid B \cap C) = \RSP(A \mid C)$$

\subsection{一组事件的独立性}

设$A_1, \cdots, A_n$为$n$个事件,满足

$$\RSP(\bigcap_{i \in S} A_i) = \prod_{i \in S} \RSP(A_i)$$

对任意子集$S$成立,则称$A_1, \cdots, A_n$为互相独立事件。

对事件$A_1, A_2, A_3$独立性条件来说,需要满足:

$$\RSP(A_1 \cap A_2) = \RSP(A_1)\RSP(A_2)$$
$$\RSP(A_1 \cap A_3) = \RSP(A_1)\RSP(A_3)$$
$$\RSP(A_2 \cap A_3) = \RSP(A_2)\RSP(A_3)$$
$$\RSP(A_1 \cap A_2 \cap A_3) = \RSP(A_1)\RSP(A_2)\RSP(A_3)$$

因此有重要的一些\kt{性质}:

\begin{ul}
  \item \kt{两两独立}并不包含独立。
  \item 等式 $\RSP(A_1 \cap A_2 \cap A_3) = \RSP(A_1)\RSP(A_2)\RSP(A_3)$不包含独立。
\end{ul}